\section{Conclusion and further work}
This project was a great opportunity for us to discover the NLP universe and to enhance our knowledge of the Python language. We also became more familiar with the graph-oriented databases and especially with the Neo4j solution and its query language Cypher.\\

We found out that NLP is a wide and really interesting topic which would be worth a more detailed study. This wasn't possible during this project and regarding our master-thesis preparation but we all agree on the fact that this subject could be a master-thesis itself.\\

We have some ideas for future work. It would be nice to further study the content of previous sentences in order to get a more consistent dialog, for instance upgrade our last version of Anna and try to find sentences composed of more than the most frequent token (the first two ones could be a nice start). We also didn't make use of the \emph{MODE} node in the XML file, this could give the user the choice to dialog with an angry Anna or a happier one. Eventually, an evaluation of Anna's answer is possible and it would be nice to use it. We could imagine to plug a Machine-Learning solution so that Anna could learn from her mistakes (irrelevant answers).
Another way we could try to explore is to represent the sentences as Markov chains and to set n-grams size to 3 or 4. This way, we could create relevant composite punchlines and be less dependant on the movie corpus.
