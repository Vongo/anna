\section{Introduction}
\subsection{Abstract}
This project is done in the context of our semester at the University of Passau which is part of our double master degree INSA Lyon Informatique (INSA IF) - Uni Passau Information und Kommunikation (IFIK).
The goal here is to make use of what we learned in the seven first weeks of the \textit{Text Mining Project} course by developing a project. To do so we can inspire us of the ideas presented in the document provided by our supervisor Behrang Qasemi Zadeh \cite{idea} \\
Our idea is to exploit movies' scripts in order to create a bot which you can speak with. This bot is called Anna in reference to the Basshunter's song Boten Anna\footnote{Boten Anna on YouTube : https://www.youtube.com/watch?v=RYQUsp-jxDQ}. \\
This report will be organized as follows. First we will describe our motivations, then present the chosen dataset as well as some enrichments and statistics and finally we will describe how our solution works.
% 
\subsection{Motivations}
We first chose this course in order to discover a new programming language (Python) as well as a new domain of IT : Natural Language Processing.\\
NLP is an important IT domain nowadays. More and more NLP-based applications are developed and there are many different use cases. The simplest one is in our pocket since every smartphone now supports predictive text, Apple's SIRI and Android's \say{Ok Google} understand pretty much everything the user is saying and handwriting recognition is also a thing. All of this is provided by technologies based on NLP and we can safely say that, in our multilingual society, these technologies will soon be a must-have.\\
Regarding our project's topic, we all are movies aficionados and the idea of interacting with a bot using our favorite movies' punchlines was very tempting. This is also the occasion to learn more about the english language (grammatically and syntactically speaking) which isn't our mother tongue.

\subsection{Acknowledgments}
We would like to thank our supervisor Behrang for his precious help, ideas and advices during the whole project phase.\\
A special \say{Thank you !} goes to Rafael E. Banchs from the Institute for Infocomm Research of Singapore who authorized us to use his movie dialogue corpus \textit{Movie-DiC}.