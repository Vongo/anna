\definecolor{dkgreen}{rgb}{0,0.6,0}
\definecolor{gray}{rgb}{0.5,0.5,0.5}
\definecolor{mauve}{rgb}{0.58,0,0.82}
\definecolor{gray}{rgb}{0.4,0.4,0.4}
\definecolor{darkblue}{rgb}{0.0,0.0,0.6}
\definecolor{lightblue}{rgb}{0.0,0.0,0.9}
\definecolor{cyan}{rgb}{0.0,0.6,0.6}
\definecolor{darkred}{rgb}{0.6,0.0,0.0}

\lstset{
  basicstyle=\ttfamily\footnotesize,
  columns=fullflexible,
  showstringspaces=false,
  numbers=left,                   % where to put the line-numbers
  numberstyle=\tiny\color{gray},  % the style that is used for the line-numbers
  stepnumber=1,
  numbersep=5pt,                  % how far the line-numbers are from the code
  backgroundcolor=\color{white},      % choose the background color. You must add \usepackage{color}
  showspaces=false,               % show spaces adding particular underscores
  showstringspaces=false,         % underline spaces within strings
  showtabs=false,                 % show tabs within strings adding particular underscores
  frame=none,                   % adds a frame around the code
  rulecolor=\color{black},        % if not set, the frame-color may be changed on line-breaks within not-black text (e.g. commens (green here))
  tabsize=2,                      % sets default tabsize to 2 spaces
  captionpos=b,                   % sets the caption-position to bottom
  breaklines=true,                % sets automatic line breaking
  breakatwhitespace=false,        % sets if automatic breaks should only happen at whitespace
  title=\lstname,                   % show the filename of files included with \lstinputlisting;
                                  % also try caption instead of title  
  commentstyle=\color{gray}\upshape
}


\lstdefinelanguage{XML}
{
  morestring=[s][\color{mauve}]{"}{"},
  morestring=[s][\color{black}]{>}{<},
  morecomment=[s]{<?}{?>},
  morecomment=[s][\color{dkgreen}]{<!--}{-->},
  stringstyle=\color{black},
  identifierstyle=\color{lightblue},
  keywordstyle=\color{red},
  % morekeywords={xmlns,xsi,noNamespaceSchemaLocation,type,id,x,y,source,target,version,tool,transRef,roleRef,objective,eventually}% list your attributes here
}

\lstdefinelanguage{json}{}

\section{Dataset}
This section describe the chosen dataset and explain how we enriched it. Statistics about data are also presented.
\subsection{Initial Dataset}
\subsubsection{Description}
The first challenge we had to face was finding a corpus of movies' dialogues. Luckily, our supervisor provided us one called Movie-DiC. This corpus has been built by R. Banchs \cite{banchs} for research and development purposes.\\
Movie-DiC comprises 132,229 dialogues, containing 764,146 turns (also called utterances) extracted from 753 different movies. In \cite{banchs}, the author mention some basic statistics about the corpus :
\begin{itemize}
\item Avg. amount of dialogues per movie : 175.60
\item Avg. amount of utterances per dialogue : 5.78
\end{itemize}
The corpus consist on a XML document of 2,249,053 lines, its structure isn't so complex since his maximal depth is only of 3 (movie - dialogue - utterance). See Code~\ref{code:dial} for an example of a dialogue unit.\\
\lstinputlisting[language=XML,caption=Dialogue unit,label={code:dial}]{dialUnit.xml}
This dialogue is pulled out the \say{xXx} movie\footnote{xXx on IMDB : http://www.imdb.com/title/tt0295701/} produced in 2002 by Rob Cohen.
\subsubsection{Remarks}
This dataset is really interesting but some details rose small issues during the project.\\
First, as you can see above in Code~\ref{code:dial}, one dialogue is in fact a succession of tags which follows a specific patern :
\begin{enumerate}
\item Speaker : who is speaking ?
\item Mode : is he happy/suspicious/\dots ?
\item Context : description of the shot
\item Utterence : the speaker's speech
\end{enumerate}
We found it strange that all those 4 tags aren't childs of another one, let say \say{utterance}. It would make more sense given the XML spirit and would also be easier to parse. Here we had to iterate on the dialogue's childs and group every group of 4 tags together in a single entity.\\
Then, some of the movies' titles are modified. For instance \say{X Files Fight the Future The} in fact refers to the movie \say{The X Files Fight the Future}. These little modifications are a problem when it comes to the enrichments phase (see section~\ref{ssec:enrich}). \\
Finally, the XML document wasn't totally correct, there were special characters such as \say{\&} which blocked the python XML parser. A root tag was also missing.
\subsection{Enrichments}
\label{ssec:enrich}
The Movie-DiC corpus only contains data about the content of the movie, but no data about the movie itself apart from its title. In order to build statistics, presented in section \ref{ssec:stats}, we enriched the corpus with the help of the Open Movie Database (short OMDb) API\footnote{http://www.omdbapi.com/}. This API is maintained by its users and isn't endorsed by or affiliated with the more famous IMDB website\footnote{http://www.imdb.com/}.\\
The use of the OMDb is really simple. It provides a web service which allows us to search for a movie given its title. It then returns lot of data about the movie such as its released year/date, genre, director's and actor's names, language, country. It also pull out some data from imdb such as the number of votes or the average rating.\\
Here's an example of usage :\\
Query : http://www.omdbapi.com/?t=Whiplash\&plot=short\&r=json\\
Answer :
\lstinputlisting[language=json,caption=OMDb answer,label={code:omdb}]{whiplash.json}
\subsection{Statistics}
\label{ssec:stats}